\documentclass[12pt]{article}
\usepackage[a4paper,
	left=1in,
	right=1in,
	top=1in,
	bottom=1in]{geometry}
\usepackage{gensymb}
\usepackage{amsmath}
\usepackage{cancel}
\usepackage{xcolor}
\usepackage{amssymb}
\usepackage{graphicx}
\usepackage{systeme}
\usepackage{hyperref}
\usepackage{pgfplotstable,filecontents}
\pgfplotsset{compat=1.9}
\usepackage[scaled=0.9]{DejaVuSansMono}
\usepackage{listings}
\usepackage{subcaption}
\usepackage{setspace}
\usepackage{relsize}
\usepackage{bigints}
\usepackage{lscape}
\usepackage{caption}
\usepackage{indentfirst}
\usepackage{amsmath}
\usepackage{sectsty}
\usepackage{textcomp}
\usepackage{gensymb}
\usepackage{mathtools, cuted}
\usepackage{subcaption}
\usepackage{csquotes}
\usepackage{setspace}

\renewcommand\arraystretch{1.5}
\renewcommand{\thefootnote}{\fnsymbol{footnote}}
\sectionfont{\fontsize{12}{15}\selectfont}
\subsectionfont{\fontsize{12}{15}\selectfont}
\subsubsectionfont{\fontsize{12}{15}\selectfont}

\definecolor{codegreen}{rgb}{0,0.6,0}
\definecolor{codegray}{rgb}{0.5,0.5,0.5}
\definecolor{codepurple}{rgb}{0.58,0,0.82}
\definecolor{backcolour}{rgb}{0.95,0.95,0.92}

\sloppy
\renewcommand{\rmdefault}{ptm}

\lstdefinestyle{mystyle}{
	backgroundcolor=\color{backcolour},   
	commentstyle=\color{codegreen},
	keywordstyle=\color{magenta},
	numberstyle=\tiny\color{codegray},
	stringstyle=\color{codepurple},
	basicstyle=\ttfamily\scriptsize,
	breakatwhitespace=false,         
	breaklines=true,                 
	captionpos=b,                    
	keepspaces=true,                 
	numbers=left,                    
	numbersep=5pt,                  
	showspaces=false,                
	showstringspaces=false,
	showtabs=false,                  
	tabsize=4
}
\lstset{style=mystyle}

\doublespacing
\setlength\parindent{24pt}

\begin{document}
	
\section*{APPENDICES\centering}
\subsubsection*{SOURCE CODE HEIRARCHY}

The code has a layout of:

\singlespacing
\begin{lstlisting}
capstone/
`--rfid/
|  `--main.cpp
`--src/
|  `--__init__.py
|  `--bin/
|  |  `--__init__.py
|  |  `--access.py
|  |  `--code_email.py	
|  `--misc/
|  |  `--__init__.py
|  |  `--colors.py
|  `--system/
|  |  `--utils/
|  |  |  `--password_gen.py
|  |  |  `--update.py
|  |  `--.bashrc
|  |  `--pre_setup.sh
|  |  `--user_setup.sh
|  `--cli.py
|  `--face_recog.py
|  `--function.py
|  `--main.py
`--test_faces/
|  `--biden1.jpg
|  `--biden2.jpg
|  `--f0.jpg
|  `--f1.jpg
|  `--f2.jpg
|  `--f3.jpg
|  `--f4.jpg
|  `--f5.jpg
|  `--f6.jpg
|  `--f7.jpg
|  `--f8.jpg
|  `--f9.jpg
|  `--obama1.jpg
|  `--obama2.jpg
`--sample/
|  `--test1.png
|  `--test2.jpg
|  `--test2.png
`--README.md
`--command_logs
`--logs
`--requirements.txt
`--.gitignore
\end{lstlisting}
\doublespacing

The main folder is \texttt{capstone}, hosted on GitHub with repository link of: \texttt{https://github.com/testno0/capstone}.

It is also made up of several branches for development of each feature and function, which are: (1.) \texttt{main} as the \texttt{main} branch; (2.) \texttt{devel} for further development of new features introduced in their temporary branch; (3.) \texttt{cli} for integration of commandline interface (CLI); (4.) \texttt{test-v1-beta} as a beta testing branch for each new features introduced before merging with \texttt{devel}; (5.) \texttt{face\_recog} for introduction of face recognition algorithms; (6.) \texttt{pep8-adaptation} for shifting of coding style to Python Enhancement Proposal (PEP) 8; (7.) \texttt{pep8-ch-test} for testing of changes in \texttt{pep8-adaptation}, since there are possibilities that the changes cause code failure; (9.) and lastly \texttt{v1.0}, for the final version.

The folder \texttt{rfid/} contains the source code for Arduino in locker system, while the folder \texttt{src/} contains the code for face recognition system. And \texttt{test\_faces/} and \texttt{samples/} contains the faces for initial testing, and live testing, respectively.

\subsection*{Installation}

The product of this project is a working-out-of-the-box program in a plug-and-play USB A stick for boot in most of systems. However, for redeployment, or perhaps a reinstallation, the program can be easily configured with the following steps.

Clone the repository with \texttt{git}:

\begin{lstlisting}
git clone https://github.com/testno0/capstone
\end{lstlisting}

And change the permission of the setup script with \texttt{chmod +x \$HOME/capstone/src/system/pre\_setup.sh} and execute it with \texttt{./\$HOME/capstone/src/system/pre\_setup.sh} or skip the setups and do \texttt{bash \$HOME/capstone/src/system/pre\_setup.sh}.

The setup script would then automatically automate the other processes, including the user setup, but would later prompt input from user for necessary information.

\subsection*{RFID locker}

\singlespacing
\begin{lstlisting}[language=C++, caption={\texttt{C++} code of the RFID locker}]
#include <LiquidCrystal_I2C.h>
#include <SPI.h>
#include <MFRC522.h>

#define SS_PIN 10
#define RST_PIN 9

String UID = "43 12 9F 94", UID_2 = "B3 E5 F 95";
bool lock = true, check;

Servo servo;
LiquidCrystal_I2C lcd(0x27, 20, 4);
MFRC522 rfid(SS_PIN, RST_PIN);


void lcd_print(String message, int x, int y, bool clear) {
	if (clear == true) {
		lcd.clear();
	}
	
	lcd.setCursor(x, y);
	lcd.print(message);
	
}

bool checker(bool init_) {
	if (init_ == true) {
		if ( ! rfid.PICC_IsNewCardPresent() || ! rfid.PICC_READ_CARD_SERIAL()) {
			return false;
		} 
	} else {
		if ( ! rfid.PICC_IsNewCardPresent() || ! rfid.PICC_READ_CARD_SERIAL()) {
			return false;
		} else {
			return true;
		}
	}
	
}


void setup() {
	Serial.begin(9600);
	servo.write(70);
	lcd.init();
	lcd.backlight();
	servo.attach(3);
	SPI.begin();
	rfid.PCD_Init();
	
}

void loop() {
	lcd_print("Welcome!", 6, 1, false);
	lcd_print("Please scan the card", 0, 2, false);
	check = checker();
	
	if (check == false) {
		return;
	} 
	
	
	lcd_print("Scanning", 0, 0, true);
	lcd_print("Please wait", 0, 1, false);
	Serial.print("UID tag is: ");
	
	String ID = "";
	for (byte i = 0; i < rfid.uid.size(); i++) {
		lcd.print(".");
		ID.concat(String(rfid.uid.uidByte[i] < 0x10? " 0" : " "));
		ID.concat(String(rfid.uid.uidByte[i], HEX));
		delay(300);
		
	}
	
	ID.toUpperCase();
	
	
	// start rfid card recognition
	if (ID.substring(1) == UID || ID.substring(1) == UID_2) {
		servo.write(160);
		lock = false;
		
		while (lock != true) {
			if (lock == false) {
				break;
				
			}
			
			lcd_print("Door opened.", 4, 1, true);
			delay(1500);
			check_2 = checker();
			if (check == true) {
				lock = false;
				servo.write(70);
				continue;
				
			}
			
		}
	} else {
		lcd_print("Access denied.", 4, 1, true);
		
	}
	
}
\end{lstlisting}
\doublespacing

\subsection*{RFID-based attendance system}

The code has a layout of, where as \texttt{bin/} contains the modules for execution of \texttt{main.py}, \texttt{misc/} less unimportant module, \texttt{system/} contains the utilities of the program as well as the setup scripts.

\singlespacing
\begin{lstlisting}
`--src/
|  `--__init__.py
|  `--bin/
|  |  `--__init__.py
|  |  `--access.py
|  |  `--code_email.py	
|  `--misc/
|  |  `--__init__.py
|  |  `--colors.py
|  `--system/
|  |  `--utils/
|  |  |  `--password_gen.py
|  |  |  `--update.py
|  |  `--.bashrc
|  |  `--pre_setup.sh
|  |  `--user_setup.sh
|  `--cli.py
|  `--face_recog.py
|  `--function.py
|  `--main.py
...
\end{lstlisting}
\doublespacing

\texttt{\_\_init\_\_.py} is a file used to declare the directory as a regular packages, as suggested by Python Documentation (2022). While the \texttt{./misc/} directory is a regular package for \texttt{colors.py} to distinguish the \texttt{stdout} and \texttt{stdin} of the program from other \texttt{stderr}.

\subsubsection*{System Setup}

Since the device aims to be easily deployable, and easy to use. The project authors decided to device a setup script written in \texttt{shell}, since this is mainly for *NIX systems. First was the \texttt{pre\_setup.sh} for system setup, which would be executed after permission change with command of: \texttt{chmod +x pre\_setup.sh}.

\singlespacing
\begin{lstlisting}[language=bash, caption={\texttt{pre\_setup.sh} for system setup on \texttt{rpm}-based Linux distributions.}]
# system upgrade: #* pass
echo -e "\e[1;32m> Starting full system upgrade to fix CVE vulnerabilities ...\e[0m\nKindly input the current password : 'root' (no quotations) when prompted, and don't let the system die."
sudo dnf update -y

# check for system utilities: #? likely passing    
echo -e "\e[1;32m> Checking the presence of git, installing if not installed ...\e[0m"    
git_check=$(dnf list installed | grep -i "git")
if [[ $git_check != *"git"* ]]; then
	echo -e "\e[1;32m> Installing git in the system ...\e[0m"
	sudo dnf install git -y
fi

# install pip if not installed: #? likely passing
echo -e "\e[1;32m> Checking the presence of python.pip, installing if not installed ...\e[0m"
pip_check=$(dnf list installed | grep "python3-pip")
if [[ $pip_check != *"python3-pip"* ]]; then
	echo -e "\e[1;32m> Installing python3-pip in the system ...\e[0m"
	sudo dnf install python3-pip -y
fi

echo -e "\e[1;32m> Checking the presence of cmake, installing if not installed ...\e[0m"
cmake_check=$(dnf list installed | grep -i "cmake[^-]$")
if [[ $cmake_check != *"cmake"* ]]; then
	echo -e "\e[1;32m> Installing python3-pip in the system ...\e[0m"
	sudo dnf install cmake -y
fi

echo -e "\e[1;32m> Checking the presence of cmake, installing if not installed ...\e[0m"
dlib_check=$(dnf list installed | grep -i "python.-dlib")
if [[ $dlib_check != *"python3-dlib"* ]]; then
	echo -e "\e[1;32m> Installing python3-pip in the system ...\e[0m"
	sudo dnf install python3-dlib -y
fi

# install required packages: #? likely passing
echo -e "\e[1;31m> Installing required packages ...\e[0m"
pip install -r $HOME/capstone/requirements.txt

# setup a systemd service for repository check : #! FAILED
echo -e "\e[1;32m> Setting up a systemd service ...\e[0m"
sudo touch /etc/systemd/system/repository-check.service
echo -e "[Unit]\nDescription=Check the repository for updates every 24 hours.\nAfter=network.target\nStartLimitIntervalSec=5\n\n[Service]\nType=simple\nRestart=always\nRestartSec=5\nUser=\"%u\"\nExecStart=/usr/bin/env python \"%h\"/repository/bin/service.py'\n\n[Install]\nWantedBy=multi-user.target" | sudo tee -a /etc/systemd/system/repository-check.service

# replaced with "specifiers" as described from systemd documentation, refer to: https://www.freedesktop.org/software/systemd/man/systemd.unit.html#Specifiers



#! FAILED

sudo systemctl daemon-reload
sudo systemctl enable repository-check.service

# setup the dirs
mkdir -P $HOME/.att_sys/bak

cp --recursive $HOME/capstone/src -t $HOME/.att_sys
mv -recursive $HOME/capstone -t $HOME/.att_sys/bak

# move the binaries: #? likely passing
# main binaries@$HOME/.att_sys/bin/
# cp $HOME/capstone/src/bin/*.py $HOME/.att_sys/bin/

# algorithm@$HOME/.att_sys/algorithm/
# cp $HOME/capstone/src/*.py $HOME/.att_sys/

# utils@$HOME/.att_sys/system/utils
# cp $HOME/capstone/src/system/utils/*.py $HOME/.att_sys/system/utils
# download the setup script instead of moving it
# pre setup script
# wget https://raw.githubusercontent.com/testno0/capstone/devel/src/system/pre_setup.sh -P $HOME/.att_sys/system/
# user setup script
# wget https://raw.githubusercontent.com/testno0/capstone/devel/src/system/user_setup.sh -P $HOME/.att_sys/system/

# misc@HOME/.att_sys/misc
# cp $HOME/capstone/src/misc/*.py $HOME/.att_sys/misc
# cp $HOME/capstone/requirements.txt $HOME/.att_sys/

# remove the old bashrc
rm $HOME/.bashrc
wget https://raw.githubusercontent.com/testno0/capstone/devel/src/system/.bashrc -P $HOME/
source $HOME/.bashrc

chmod -R +x $HOME/.att_sys/system/*.sh
./$HOME/.att_sys/system/user_setup.sh

\end{lstlisting}
\doublespacing

The script is mainly an automatic to perform system update, as well as installation of important packages such as \texttt{git} for repository access, \texttt{cmake} and \texttt{python3-dlib} as depedencies of dependencies in \texttt{requirements.txt}, specifically by \texttt{face\_recognition} module, and \texttt{python3-pip} was then used for installation of python dependencies. It also includes the automated installation of program dependencies, and setup of the \texttt{systemd} service, to be able to schedule the use of the module to avoid memory leak and overuse:

\singlespacing
\begin{lstlisting}[caption={\texttt{systemd} service written to intercept pull an update to repository every 24 hours.}]
	[Unit]
	Description=Check the repository for updates every 24 hours.
	
	[Service]
	Type=simple
	ExecStart=python ~/repository/bin/service.py
	
	[Install]
	WantedBy=multi-user.target
\end{lstlisting}
\doublespacing

After the setup of \texttt{systemd} service, the source code was then moved into \texttt{\$PATH} defined in \texttt{.bashrc}. Whereas the \texttt{.bashrc}, also mentioned in line 79, 80, and 81, is the modified BASH configuration from default of Fedora Linux 35, that contains:

\singlespacing
\begin{lstlisting}[language=bash, caption={\texttt{.bashrc}}]
# .bashrc

# Source global definitions
if [ -f /etc/bashrc ]; then
	. /etc/bashrc
fi

# User specific environment
if ! [[ "$PATH" =~ "$HOME/.local/bin:$HOME/bin:" ]]
then
	PATH="$HOME/.local/bin:$HOME/bin:$PATH"
fi
export PATH

# Uncomment the following line if you don't like systemctl's auto-paging feature:
# export SYSTEMD_PAGER=

# User specific aliases and functions
if [ -d ~/.bashrc.d ]; then
	for rc in ~/.bashrc.d/*; do
		if [ -f "$rc" ]; then
		. "$rc"
		fi
	done
fi

PS1="\u@\h:\w $"
export PS1

PROMPT_DIRTRIM=2

alias taptap="python $HOME/.att_sys/cli.py"

unset rc
\end{lstlisting}
\doublespacing

And finally, for the last phase of the system setup, is the user setup, this includes the collection of details, for use of the Python modules, this includes email address for messages, name, the name of the school, and the creation of 32 character (\textit{pseudo-random}) ASCII string passphrase, using \texttt{./utils/password\_gen.py}:

\singlespacing
\begin{lstlisting}[language=bash, caption={\texttt{user\_setup.sh}}]
# test 1: #* passed!

if [ ! -d "$HOME/.att_sys/" ]; then
	# user information setup : #* pass
	echo -e "\e[1;32m> User setup ...\e[0m"
	
	# take the username and email of the user
	echo -e "\e[1m> This information is to be used in database, kindly use correct punctuation.\e[0m"
	read -p "> Enter the email for access codes and reports: " email
	read -p "> Enter the username: " user_name
	read -p "> Enter the school name: " school_name
	
	# generate user password
	password=$(python $HOME/.att_sys/system/utils/password_gen.py)
	echo -e "\e[1;32m> Your password is: $password"
	
	# append user information and password to a file
	echo -e "\e[1;32m> Appending user info ...\e[0m"
	echo -e "$email\n$user_name\n$password\n$school_name" > $HOME/.att_sys/user_info
	
	python $HOME/.att_sys/system/utils/update.py
	
	sec=10
	while [ $sec -ge 0 ]; do
		echo -e "\e[32m> Setup done! Rebooting after :" $sec"s ...\e[0m\r" 
		let "sec=sec-1"
		sleep 1
	done
	
	systemctl poweroff

else
	#* passed
	echo -e "\e[1;31m> System already setup ...\e[0m"

fi
\end{lstlisting}

\singlespacing
\begin{lstlisting}[language=Python, caption={\texttt{user\_setup.sh}}]
import random
import string

str_set = [
	string.ascii_lowercase,
	string.ascii_uppercase,
	string.punctuation,
	string.digits
]

print(''.join([random.choice(random.choice([str_set])) for x in range(32)]))
\end{lstlisting}
\doublespacing

Instead of turning it into function, its \texttt{stdout} was instead passed on directly as \texttt{stdin} for line 19, in \texttt{user\_setup.sh}: \texttt{password=\$(python \$HOME/.att\_sys/system/utils/password\_gen.py)}. This is done since BASH variables can't receive Python's \texttt{return} as \texttt{stdin}.

\subsubsection*{Algorithm and the program}

Since the algorithm pulls an update every 24 hours, automatically, from the \texttt{git} repository, a persistent module was created that will pull an update from the repository and return the changes to the current database. This was done using \texttt{os.system()} function from \texttt{Python}'s standard library.

\singlespacing
\begin{lstlisting}[language=Python, caption={Python algorithm for the \texttt{systemd} service to be used to pull a git update on repository.}]
def pull_data(self):
	try:
		if exists(f"{self.HOME}/capstone"):
			sys(f"rm -rf {self.HOME}/capstone")
			sys(f"git clone --branch database {self.repo}")
			
			return True
	except SystemError or KeyboardInterrupt or OSError or ConnectionError:
		return False

def get_data(self):
	count = 0
	
	while count < 3:
		try:
			with open(f"{self.HOME}/repo/<filename>") as data:
				studentDATA = json.load(data)
			
			with open(f"{self.HOME}/repo/<filename>") as Data:
				teacherDATA = json.load(Data)
			
			return studentDATA, teacherDATA
		
		except FileNotFoundError:
			self.pull_data()
			count += 1
			continue
		else:
			raise SystemExit(f"""\
				{C.BOLD+C.RED}> Too much error, please try again later.{C.END}
			""")

\end{lstlisting}
\doublespacing

Which was further coupled with other functions, such as \texttt{av\_cams()} and \texttt{setup()}, to check for available cameras for use in \texttt{face\_recog.py}, and repository setup for \texttt{main.py}, respectively. As a whole, the code reads as:

\singlespacing
\begin{lstlisting}[language=Python, caption={\texttt{function.py}}]
import json
import cv2 as cv

from os import system as sys
from os.path import exists

from bin.access import access
from misc.colors import colors as C


def av_cams():
	index, arr = 0, []
	
	while True:
		cap = cv.VideoCapture(index)
	
	if not cap.read()[0]:
		break
	else:
		arr.append(index)
		cap.release()
		index += 1
	
	if arr == []:
		raise SystemExit(f"{C.RED+C.BOLD}> No camera available.{C.END}")
	else:
		input(f"""
			{C.GREEN+C.BOLD}> All available cameras: {[f'{num} {cam}' for num, cam in enumerate(arr)]}{C.END}\nPress any key to clear ...
		""")
		sys.stdout.write("\033[K")
	

class System:

	def __init__(self, HOME, repo, admin_email):
		self.HOME = HOME
		self.repo = repo
		self.admin_email = admin_email
	
	def pull_data(self):
		try:
			if exists(f"{self.HOME}/capstone"):
				sys(f"rm -rf {self.HOME}/capstone")
				sys(f"git clone --branch database {self.repo}")
		
				return True
		except SystemError or KeyboardInterrupt or OSError or ConnectionError:
			return False
		
	def get_data(self):
		count = 0
		
		while count < 3:
			try:
				with open(f"{self.HOME}/repo/<filename>") as data:
					studentDATA = json.load(data)
				
				with open(f"{self.HOME}/repo/<filename>") as Data:
					teacherDATA = json.load(Data)
				
				return studentDATA, teacherDATA
			
			except FileNotFoundError:
				self.pull_data()
				count += 1
				continue
			else:
				raise SystemExit(f"""\
					{C.BOLD+C.RED}> Too much error, please try again later.{C.END}
				""")
		
	
	def setup(self, school_name):
		ret = access()
		trial = 0
		
		if not ret:
			raise SystemExit(f"""\
				{C.BOLD+C.RED}> Too much error, please try again later.{C.END}
			""")
		else:
			try:
				while True:
					if trial == 3:
						break
					
				if not self.pull_data():                    
					trial += 1
					continue
				else:
					break
			except KeyboardInterrupt:
				raise SystemExit(f"""\
					{C.BOLD+C.RED}> Too much error, please try again later.{C.END}
				""")
			else:
				return self.get_data()
		
\end{lstlisting}
\doublespacing

Then a module for authorized access was created, these are \texttt{access.py} to use the 32 character \textit{pseudo-random} ASCII string security code generated and emailed by \texttt{code\_email.py}, for access in local database, as well as re-setup of new administrator.

\singlespacing
\begin{lstlisting}[language=Python, caption={\texttt{access.py}}]
import os

from misc.colors import colors
from bin.code_email import Email


HOME = os.path.expanduser('~')
C = colors()


def access():
	with open(f"{HOME}/.att_sys/user_info") as info:
		source = info.readlines()
	
	receiver_email, user = source[0].rstrip().strip(), source[1].rstrip().strip() 
	password, school_name = source[2].rstrip().strip(), source[3].rstrip().strip()
	email = Email(receiver_email, user)
	
	try:
		trial, mark = 0, False
		
		while True:                  
			if trial == 3:
				try:
					email.send("alert", school_name)
				except ConnectionError:
					print(f"{C.BOLD+C.RED}> Connection error.{C.END}")
				finally:
					print(f"{C.BOLD+C.RED}> Too much error. Signing off.{C.END}")
					os.system("systemctl poweroff")
				
			if mark:
				verify = input(f"{C.BOLD}> Kindly input your 32 character password (case sensitive {3-trial} left): {C.END}")
				
				if verify != password:
					trial += 1
					send_new = input(f"""\
						{C.RED+C.BOLD}> Password doesn't match. {3-trial} left.{C.END}\n{C.BOLD}Send a new temporary password to your email instead? [y/N]: {C.END}
					""")
					
					if send_new in ['y', 'Y']:
						mark = True
					
					continue
				else:
					return True
				
			else:
				new_pass = email.send("setup", school_name)
				verify_new = input(f"""\
					{C.BOLD}> Kindly input your 32 character password (case sensitive {3-trial} left): {C.END}
				""")
				
				if verify_new != new_pass:
					print(f"{C.RED+C.BOLD}> Password doesn't match. {3-trial} left.{C.END}")
					trial += 1
					continue
				else:
					return True
				
	except KeyboardInterrupt:
		print(f"{C.BOLD+C.RED}> Probable intruder. Signing off.{C.END}")   
		os.system("systemctl poweroff")
\end{lstlisting}
\doublespacing 

The code above gives 3 strict tries for user. The first try requires the locally generated passphrase during user setup, however if the user failed to authenticate, the program would prompt an \texttt{stdin} to email the given address a new passphrase for access. And if the user still fails to authenticate, the system would enter a self destruct mode, removing the local student and teacher data, while alerting the registered email, which is done by the module below:

\singlespacing
\begin{lstlisting}[language=Python, caption={\texttt{code\_email.py}}]
import random
import string
import smtplib
import ssl
import socket
import os
import geocoder

from datetime import datetime
from email.message import EmailMessage


class Email:

	port, smtp_server = 465, "smtp.gmail.com"
	sender_email, password = "clydebotrfid@gmail.com", "CCSHSRFIDG5" # fill up later
	
	def __init__(self, receiver_email, user):
		self.receiver_email = receiver_email
		self.user = user
	
	def send(self, access, school_name, student_name=None):
		str_set = [
			string.ascii_lowercase,
			string.ascii_uppercase,
			string.punctuation,
			string.digits
		]
	
		msg, msg["From"], msg["To"] = EmailMessage(), self.sender_email, self.receiver_email
		phrase = ''.join([random.choice(random.choice(str_set)) for x in range(32)])
		
		if access == "setup": 
			msg["Subject"] = "Secure access phrase"
			msg.set_content(f"""\
			<!DOCTYPE html>
			<html>
				</body>
					<p align=justify>
						Hello, TapTap is here to deliver the secure passphrase requested by: <b>{self.receiver_email} ({self.user}) from {school_name}.</b>,<br><br>
						<em>DO NOT SHARE THIS CODE FOR USE IN SETUP/GIT PULL OF THE DATABASE</em>. This is the secure phrase for your access of the student and teacher data in repository setup: <br><br><center><b><code>{phrase}</code></b></center>
					</p>
				
					<p align=justify> 
						If you are not trying to access, ignore this email. The details of the computer are:<br><br>User: <i>{os.getlogin()}</i><br>HOSTNAME: <i>{socket.gethostname()}</i><br>IP address: <i>{socket.gethostbyname(socket.gethostname())}</i><br>Tracked from: <i>{geocoder.ip('me')}) at {datetime.now().strftime('%d/%m/%Y %H:%M:%S')}</i>.<blockquote>Cheers,<br>TapTap team</blockquote>
					</p>
				</body>
				</html>
				""", subtype="html")    
				
			elif access == "alert":
				msg["Subject"] = "Breach attempt alert."  
				msg.set_content(f"""\
				<!DOCTYPE html>
				<html>
					</body>
						<p align=justify>
							Hello, TapTap is here to alert <b>{self.receiver_email} ({self.user}) from {school_name} of possible breach alert.</b><br><br>
							Your device has been receiving various passphrase mismatch. Due to the setup of device, it is shutting down as you receive this email. You can access it later on.
						</p>
					
						<p align=justify> 
							If you are not trying to access, ignore this email. The details of the computer are:<br><br>User: <i>{os.getlogin()}</i><br>HOSTNAME:
							<i>{socket.gethostname()}</i><br>IP address: <i>{socket.gethostbyname(socket.gethostname())}</i><br>Tracked from:<i> {geocoder.ip('me')}) at {datetime.now().strftime('%d/%m/%Y %H:%M:%S')}</i>.
						</p>
					
						<p align=justify>
							This was done as security measure, the email in situation that met the criterion is inevitable.<br><blockquote>Cheers,<br>TapTap team</blockquote>
						</p>
					</body>
				</html>
				""", subtype="html")
			
		elif access == "student true":
			msg["Subject"] = f"{student_name} registered"
			msg.set_content(f"""\
			<!DOCTYPE html>
			<html>
				</body>
					<p align=justify>
						Hello, TapTap is here to notify that {student_name} of {school_name} has finally arrived his class at {datetime.now().strftime('%d/%m/%Y %H:%M:%S')}, and is present in school today. Thank you!
					</p>
				
				</body>
			</html>
			""", subtype="html")            
		
		try:        
			# send the email to the user for the code.         
			context = ssl.create_default_context()
			with smtplib.SMTP_SSL(self.smtp_server, self.port, context=context) as server:    
				server.login(self.sender_email, self.password)
				server.send_message(msg)
		except ConnectionError:
			return "Please try again later."        
		else:	
			return phrase

\end{lstlisting}
\doublespacing

\texttt{access.py} then was aso used in \texttt{main.py}, as well as the update utility residing in \texttt{./system/utils/update.py} that performs database updates:

\singlespacing
\begin{lstlisting}[language=Python, caption={\texttt{code\_email.py}}]
import os

from src.bin.access import access
from misc.colors import colors as C


HOME = os.path.expanduser('~')


def update():
	ret = access()
	
	if ret:
		if not os.path.exists(f"{HOME}/repo"):        
			os.system("git clone https://github.com/testno0/repo $HOME/ &> /dev/null")
		else:
			print(f"{C.BOLD+C.GREEN} User system setup passed.{C.END}")
			os.system(f"cd {HOME}/repo/ && git pull")
	else:
		# false phase, pass : #? passing
		os.system("rm -rf {HOME}/repo/")
		print(f"""
			{C.BOLD+C.RED}> Verification error, repository was nuked by system for security.{C.END}
		""")            
		os.system("systemctl poweroff")
		
\end{lstlisting}
\doublespacing

\texttt{main.py} on the other hand was the main module by the program, this is called by \texttt{cli.py}, which is the interface that would provide the user interaction with the system, for general use:

\singlespacing
\begin{lstlisting}[language=Python, caption={\texttt{code\_email.py}}]
from sys import argv as INPT, stdout
from difflib import SequenceMatcher as SM
from os import path
from time import sleep

from function import System
from bin.code_email import Email
from misc.colors import colors as C

# repo link is the link of this repository https://github.com/testno0/repo
# although leave it blank as first, no one touchers the parameters


def main(school_name=school_name, source=source):
	"""Initiate the system, use try, except, else block to catch errors and to organize the procedures based on the cases the system gives."""

	HOME = path.expanduser('~')
	
	with open(f"{HOME}/.att_sys/user_info") as info:
		source = info.readlines()
	
	receiver_email, user = source[0].rstrip().strip(), source[1].rstrip().strip() 
	password, school_name = source[2].rstrip().strip(), source[3].rstrip().strip()
	sysINIT = System(HOME, "https://github.com/testno0/repo", receiver_email)

	try:
		print(f"{C.GREEN+C.BOLD}> Fetching data.{C.END}")
		if not path.exists(f"{HOME}/repo"):
			print(f"""\
				{C.GREEN+C.BOLD}> The repository is not setup. Setting up the repository.{C.END}
			""")
			studentDATA, teacherDATA = sysINIT.setup(school_name)        
		else:
			studentDATA, teacherDATA = sysINIT.get_data()
	except ConnectionError: # add other exceptions later
		raise SystemExit(f"{C.RED+C.BOLD}> Connection Error.{C.END}")
	except KeyboardInterrupt:
		raise SystemExit(f"{C.RED+C.BOLD}> Keyboard Interrupt.{C.END}")
	except SystemError:
		raise SystemExit(f"{C.RED+C.BOLD}> System Error.{C.END}")
	else:
		# notify the user
		print(f"{C.GREEN+C.BOLD+C.BLINK}> System ready.{C.END}", end="\r")
		
		# init free time
		sleep(5)
		# remove the messages
		stdout.write("\033[K")
		
		email = Email()
		while True:
			cardID = INPT[0]            
			
			for ID in studentDATA:
				if SM(None, cardID, ID).ratio() == 1:
					print(f"{C.GREEN+C.BOLD}> Student recognized.{C.END}")
					email.send("student true", source[3].rstrip().strip(), studentDATA[ID][0])
				else:
					print(f"{C.RED+C.BOLD}> Error.{C.END}")
				# leave at blank first
					email.send("student true", source[3].rstrip().strip(), studentDATA)[ID][0]
			continue
\end{lstlisting}
\doublespacing

And for the face recognition:

\singlespacing
\begin{lstlisting}[language=Python, caption={\texttt{code\_email.py}}]
import face_recognition as fr
import numpy as np
import cv2 as cv

from os.path import expanduser

from functions import av_cams


def face_recognition():
	av_cams()
	path_ = f"{expanduser('~')}/temporary/capstone/sample/"
	
	# load face references from path_.
	# ezekiel lopez encoding
	ref_face = fr.load_image_file(f"{path_}/test_1.png")    
	rf_encoding = fr.face_encodings(ref_face)[0]
	
	# laisie angela donato encoding
	ref_face_2 = fr.load_image_file(f"{path_}/test_2.png")
	rf_encoding2 = fr.face_encodings(ref_face_2)[0]
	
	known_fe = [
		rf_encoding,
		rf_encoding2,    
	]
	known_fnames = [
		"Ezekiel Lopez",
		"Laisie Angela Donato",
	]
	
	# Initialize some variables
	face_locations, face_encodings, face_names = [], [], []     
	process_this_frame = True
	
	# video capture
	vid = cv.VideoCapture(0)
	
	while True:   
		# take frame and references from video capture
		_, frame = vid.read()
		
		# resizing to smaller frame, to avoid much larger use of gpu and 
		# memory, also to decrease the processing time convert to
		# another color.
		small_frame = cv.resize(frame, (0, 0), fx=0.25, fy=0.25)
		
		# convert to another color.
		rgb_small_frame = small_frame[:, :, ::-1]
		
		if process_this_frame:
			# get all the face endcoding and location in the current 
			# frame returned by the live camera input.
			face_locations = fr.face_locations(rgb_small_frame)
			face_encodings = fr.face_encodings(rgb_small_frame, face_locations)
			
			face_names = []
			for face_encoding in face_encodings:
				matches = fr.compare_faces(known_fe, face_encoding)
				name = "unknown"
				
				# Or instead, use the known face with the smallest
				# distance to the new face
				face_distances = fr.face_distance(known_fe, face_encoding)
				best_match_index = np.argmin(face_distances)
				if matches[best_match_index]:
					name = known_fnames[best_match_index]
				
				face_names.append(name)
			
			for (top, right, bottom, left) in face_locations:
				top *= 4
				right *= 4
				bottom *= 4
				left *= 4
				
				# output the box in the frame
				if name != "unknown":
					cv.rectangle(frame, (left, top), (right, bottom), (0, 255, 0), 2)
				
					# Draw a label with a name below the face
					cv.rectangle(frame, (left, bottom - 35), (right, bottom), (0, 255, 0), cv.FILLED)
					font = cv.FONT_HERSHEY_DUPLEX
					cv.putText(frame, name, (left + 6, bottom - 6), font, 1.0, (255, 255, 255), 1)
				else:
					cv.rectangle(frame, (left, top), (right, bottom), (0, 0, 255), 2)
				
					# Draw a label with a name below the face
					cv.rectangle(frame, (left, bottom - 35), (right, bottom), (0, 0, 255), cv.FILLED)
					font = cv.FONT_HERSHEY_DUPLEX
					cv.putText(frame, name, (left + 6, bottom - 6), font, 1.0, (255, 255, 255), 1)
					
			
			# display the resulting image
			cv.imshow("Video", frame)
		
		# hit 'q' on the keyboard to quit!
		if cv.waitKey(1) & 0xFF == ord('q'):
			break
		
	vid.release()
	cv.destroyAllWindows()
\end{lstlisting}
\doublespacing

\texttt{face\_recog.py} utilizes the \texttt{numpy}, \texttt{python-cv2}, and \texttt{face\_recognition} module, as well as other user defined functions. The \texttt{numpy} was used for calculation of face distances, \texttt{python-cv2} provides the real time access to camera input, and \texttt{face\_recognition} encodes the face of user from database as references of the real time input.

\singlespacing
\begin{lstlisting}[language=Python, caption={\texttt{code\_email.py}}]
import argparse
import sys
import os

from main import main
from function import System
from misc.colors import colors
from system.utils.update import update


def program_options():
	C = colors()
	description = """\
		This is a program designed to interact with the TapTap, an RFID system designed by Capstone
		Group 5.
	"""
	parser = argparse.ArgumentParser(
		prog="taptap",
		usage="taptap [OPTIONS]",
		description=description
	)
	
	# all of the functions listed below prompts for the local or temporary password email to the user
	# using python's stdlib smtplib and ssl, this requires internet connection, else the user need to
	# the password they received during the setup. 
	
	# start/use the system
	parser.add_argument("-use", "--use", help="Use the system.", action="store_true")
	# update the database by calling the fetch database function from src.function
	parser.add_argument("-U", "--update", help="Update the system", action="store_true")
	# (re)setup the user
	parser.add_argument(
		"-s",
		"--usersetup",
		help="Setup the user (prompts to input the passphrase sent via email, if used again).",
		action="store_true"
	)
	# destroy the system, can be used in case of intruder breach
	parser.add_argument("-d", "--destroy", help="Destroy the user database.", action="store_true")    
	
	args = parser.parse_args()
	
	if args.use:
		main() #? likely passing
	elif args.update:
		update() #* passed
	elif args.usersetup:
		os.system("./$HOME/.att_sys/system/setup.sh") #* passed
	elif args.destroy:
		os.system("echo 'rm -rf $HOME/.att_sys'") #* passed

program_options()
\end{lstlisting}
\doublespacing

And finally, is the \texttt{cli.py} that acts as mediator for all of the modules, which utilized \texttt{argparse} module for creation of program options, as well as \texttt{sys} and \texttt{os} for supporting roles.

\section*{Git and command logs}

The changes per commit/update can be viewed here, including the cryptographic hash of the change (SHA1), author of the changes\footnote[3]{iaacornus/testno0 is account of James Aaron Erang, the former is the main account, and the latter being backup account.\\\indent \indent The email were also removed for security and privacy of the author, it can be verified via \href{https://github.com/iaacornus/}{https://github.com/iaacornus/}.}, date, time, and changes done, this was recorded by version control system, \texttt{git}, and Fedora Linux 35 package of Richard Stallman's GNU BASH (Bourne Again SHell):

\singlespacing
\begin{lstlisting}[caption={\texttt{git} log of the development.}]
commit 08728ed3d6af99f7886b8fdae556b8e3ad18f553
Author: iaacornus <#########.#####@gmail.com>
Date:   Tue May 10 15:17:52 2022 +0800

	add av_cam function to simplify face_recognition.py

commit 031bb3918d0aaac6cd49384bfc2f3929fe68e69b
Author: iaacornus <#########.#####@gmail.com>
Date:   Tue May 10 15:17:14 2022 +0800

	another rfid build

commit 4948218e6cae07badc0c8a0dcbc05ff45b81182e
Author: iaacornus <#########.#####@gmail.com>
Date:   Tue May 10 15:16:54 2022 +0800

	face sample for testing

commit fa50446f592f10175c1251c894f9ea27c383fcec
Author: iaacornus <#########.#####@gmail.com>
Date:   Tue May 10 15:16:42 2022 +0800

	successful face recognition build

commit 8912de58f0fe29dbcfb980a31292cc2444093f2b
Author: iaacornus <#########.#####@gmail.com>
Date:   Tue May 10 15:16:22 2022 +0800

	face recognition pycache

commit 3c4e3230dc286259f5823b6630616e508262190e
Author: testno0 <96910593+testno0@users.noreply.github.com>
Date:   Tue May 10 12:20:55 2022 +0800

	fixed lock mechanism

commit 6e282724d21103bf438c2e87b1c74ff79fa62129
Author: testno0 <96910593+testno0@users.noreply.github.com>
Date:   Tue May 10 12:15:54 2022 +0800

	upp

commit 96edca1570255437b786fe3696a9932c6ccc31a5
Author: testno0 <96910593+testno0@users.noreply.github.com>
Date:   Tue May 10 12:01:14 2022 +0800
	
	fixed while loop

commit 19568e5ed5dc91e47dc956409b77e02d927033a0
Author: testno0 <96910593+testno0@users.noreply.github.com>
Date:   Tue May 10 11:28:47 2022 +0800

	optimized algorithm

commit 1473b0c7320b6758a2f359c690015d3cf3425093
Author: testno0 <96910593+testno0@users.noreply.github.com>
Date:   Tue May 10 10:35:55 2022 +0800

	Create main.cpp

commit c916ec11f5453df8c08dbdfce56208f55bccbfc2
Author: testno0 <##########@gmail.com>
Date:   Wed May 4 01:16:57 2022 +0800

	changed exec env to specifiers

commit f2ccbe8a081c51fcfa44dbbaaeea003b04fbe729
Author: testno0 <##########@gmail.com>
Date:   Wed May 4 00:46:00 2022 +0800

	initial codebase

commit 33dc8651f092e5e13b83630b9249160e313e2b1c
Author: testno0 <##########@gmail.com>
Date:   Wed May 4 00:45:44 2022 +0800

	add testing py file for fr

commit c7fb7431596817e7d64d9e5ba02b90d9121ac7eb
Author: testno0 <##########@gmail.com>
Date:   Wed May 4 00:14:30 2022 +0800
	
	add python image library (PIL) -- pillow as deps

commit 1a77c7460088380955888877f326eeecf9425cd1
Author: testno0 <##########@gmail.com>
Date:   Wed May 4 00:14:10 2022 +0800

	add obama faces as another ref

commit 1f1fb87451955279afd0c5eb70e098fa20cc03cf
Author: testno0 <##########@gmail.com>
Date:   Wed May 4 00:13:57 2022 +0800

	add joe biden faces as references

commit 9a0cba4fe1a186bc2a5961be460ba9568a17f225
Author: testno0 <##########@gmail.com>
Date:   Wed May 4 00:08:23 2022 +0800

	add sample faces for testing

commit de938512623ebb2c324946d23f67a906ba8d891e
Author: testno0 <##########@gmail.com>
Date:   Tue May 3 23:44:48 2022 +0800

	add back opencv2 for coupling with face_recognition module

commit 789712fa9146673f0e17d8c51013ffb1c335aa3f
Merge: 756cbaf be696d8
Author: testno0 <96910593+testno0@users.noreply.github.com>
Date:   Tue May 3 23:37:36 2022 +0800

	Merge pull request #2 from testno0/cli

	mini fixes, preperation for face recognition devel

commit be696d8d42bddcbd177afcfcdcf43870a77ea059
Author: testno0 <##########@gmail.com>
Date:   Tue May 3 23:34:00 2022 +0800

	add the command logs in the virtual env used

commit 81bc86ba2af5b0096180b6dea23a6bc012b4b3e8
Author: testno0 <##########@gmail.com>
Date:   Tue May 3 23:33:23 2022 +0800

	add the new commits

commit 1abefd59c0e92ee9d749c9d8d36febdce101cc51
Author: testno0 <##########@gmail.com>
Date:   Tue May 3 23:32:01 2022 +0800

	simplified the src code moving into the app folder

commit 2b64a6b05571fddc7f620e431d54d31fa13868bf
Author: testno0 <##########@gmail.com>
Date:   Tue May 3 23:31:30 2022 +0800

	add face_recognition as requirements and removed cv2

commit 756cbaf5d9ee453a795ba6ca6e4267b0b4d5bbe0
Merge: b6acf48 6424837
Author: testno0 <96910593+testno0@users.noreply.github.com>
Date:   Sun May 1 21:37:09 2022 +0800

	Merge pull request #1 from testno0/cli

	cli implementation and other tests from test-v1-beta merged to devel branch for further testing and coding as well as implementation of other features

commit 6424837a3bfddb1b63959a0552de4f4afe90e819
Author: testno0 <##########@gmail.com>
Date:   Sun May 1 21:33:01 2022 +0800

	the loop bug were removed, the trial limit were also barred into 3

	the feedback were also changed

commit 0d7eea85a30488ef36919ba94ee14db4e392cbe8
Author: testno0 <##########@gmail.com>
Date:   Sun May 1 21:32:23 2022 +0800

	satisfied the requirements of imports from code_email and function.py

commit 37d76b8969274a929094033b5fe70d074ccb0668
Author: testno0 <##########@gmail.com>
Date:   Sun May 1 21:30:24 2022 +0800

	tested and add commands

commit 476f4461b8cc05de87f72255b75d1b0b9bcf55df
Author: testno0 <##########@gmail.com>
Date:   Sun May 1 21:05:24 2022 +0800

	changed the message if the student is present

commit 8a91a492c273e724e230482e8f36cc0fc30a889b
Author: testno0 <##########@gmail.com>
Date:   Sun May 1 21:05:01 2022 +0800

	fixed the system structure and add chmod to the script to easily execute the script

commit 95c0ef822e1ccbb95a6532a3dd9c1219b325a96a
Author: testno0 <##########@gmail.com>
Date:   Sun May 1 21:03:55 2022 +0800

	used > redirection to avoid errors

commit fe0a08a097df30a50656fe88984d3073698b8bbf
Author: testno0 <##########@gmail.com>
Date:   Sun May 1 21:03:27 2022 +0800

	changed the dir of alias

commit 5a571de6825076bceb3b6046633dba54951c1a86
Author: testno0 <##########@gmail.com>
Date:   Sun May 1 20:49:16 2022 +0800

	add info about merging plan, and documentation of cli

commit 3d91f5f95e7582d972367f9d01bb5a13c7f10123
Author: testno0 <##########@gmail.com>
Date:   Sun May 1 20:48:48 2022 +0800

	reexproted the logs as of may 01, 2022 @ 20:48

commit f31e9cddbb5751d93f4d236b3c3f85f1e099fdb1
Author: testno0 <##########@gmail.com>
Date:   Sun May 1 20:30:38 2022 +0800

	fixed the output as well as exception handling, and variable ref

commit 71dadfe6b5feba0b2dc948cca4290a7f77eadcbc
Author: testno0 <##########@gmail.com>
Date:   Sun May 1 20:29:29 2022 +0800

	fixed the setup() function

	other information were instead retrieved from the user file info instead of being a parameter

commit d4f53bde7fa1fca58c14de78c0a925585dfc3b6a
Author: testno0 <##########@gmail.com>
Date:   Sun May 1 20:28:23 2022 +0800

	fixed imports and conditions

commit af0959f35eb4c1d4dbe560016115a4945d135f28
Author: testno0 <##########@gmail.com>
Date:   Sun May 1 20:26:55 2022 +0800

	removed other entries

commit caab29e7867b022f5a9076a6cbb315a971be8d69
Author: testno0 <##########@gmail.com>
Date:   Sun May 1 20:24:17 2022 +0800

	removed tests folder since it is nonsensical

commit 58e37cd49c8195a6e8fae9f6943382b91e2a4f5e
Author: testno0 <##########@gmail.com>
Date:   Sun May 1 20:23:40 2022 +0800

	aliased python cli.py to taptap instead of creating a setup.py script

commit a6551f4c70df4fd1722582bf44d18a32d236939b
Author: testno0 <##########@gmail.com>
Date:   Sun May 1 19:50:06 2022 +0800

	moved the user setup part from setup.sh

commit 19f03e5faa4243b37dc17291651d4a169c3cc459
Author: testno0 <##########@gmail.com>
Date:   Sun May 1 19:47:29 2022 +0800

	splited into user_setup.sh and pre_setup.sh

commit adca9b4f18cde381634c2ae6c95e2763d6064ce4
Author: testno0 <##########@gmail.com>
Date:   Sun May 1 19:47:07 2022 +0800

	moved into src/

commit 2a207b345f5b87a7f47c18e1a7e79fd858cafef0
Author: testno0 <##########@gmail.com>
Date:   Sun May 1 19:46:26 2022 +0800

	turned into function, so can be easily called from cli.py

commit 0833ccd355942f2deceaf6b001b9949bfd3aa60e
Author: testno0 <##########@gmail.com>
Date:   Sun May 1 19:45:59 2022 +0800

removed the .bashrc, since alias aren't needed anymore as the options were unified in cli.py

commit 533edcc6d6e246d22305e124ac9eca19e90f1281
Author: testno0 <##########@gmail.com>
Date:   Sun May 1 19:45:30 2022 +0800

	used for -s option in cli.py

commit 530ce099ded8022f38df86f59aa13760811878f8
Author: testno0 <##########@gmail.com>
Date:   Sun May 1 19:45:07 2022 +0800

	isolated the system setup from user setup, due to the options made in cli program

	since having to redo all the steps may brick the system and is too trivial and troublesome

commit e68e9528852c84df987d8f8cc29e5d6c7471e5de
Author: testno0 <##########@gmail.com>
Date:   Sun May 1 19:44:00 2022 +0800

	moved to src/ instead of src/algorithm

	this was done for easier navigation of the application

commit 59e7dd68fa0c1f7458480e4b72627208011a2829
Author: testno0 <##########@gmail.com>
Date:   Sun May 1 19:43:21 2022 +0800

	created cli for the program

commit dfade36107847e7643f879761d1161f041811d3c
Author: testno0 <##########@gmail.com>
Date:   Sun May 1 19:19:05 2022 +0800

	add another dir for .gitignore

commit da66dbf2e046eb786d91eb4d3fe46d889bd35f57
Author: testno0 <##########@gmail.com>
Date:   Sun May 1 19:18:05 2022 +0800

	this was added to allow relative imports

commit 215a208a96b9fad34440e4c178c906385f6f9335
Author: testno0 <##########@gmail.com>
Date:   Sun May 1 19:17:39 2022 +0800

	fixed imports

commit 1b20af51f53f08b5dffdbc87b466d7e42f08b6c8
Author: testno0 <96910593+testno0@users.noreply.github.com>
Date:   Sun May 1 18:45:01 2022 +0800

	turned into function

commit 2cb6adaa4884cb442d1f3e25dd807621abfe508c
Author: testno0 <##########@gmail.com>
Date:   Sun May 1 01:37:51 2022 +0800

	add the pycache

commit 5ad17c5c643546610cad5a4eda75a7fd5188cc6e
Author: testno0 <##########@gmail.com>
Date:   Sun May 1 01:37:13 2022 +0800

	put the loop into try, else statement to further enforce the security

commit f974397be4097fdc6d535ca2204f0ea2fd6ac2ca
Author: testno0 <##########@gmail.com>
Date:   Sun May 1 01:36:24 2022 +0800

	add alert msg, and changed some of the formatting and conducted a test

commit c96207ec81fd34b0dd01bda67cd48483b034941d
Author: testno0 <##########@gmail.com>
Date:   Sun May 1 01:36:00 2022 +0800

	add the test log of access.py

commit b6acf4811cd8df50a1c110bb6b6e46e31db95be2
Author: testno0 <##########@gmail.com>
Date:   Thu Apr 28 23:55:11 2022 +0800

	used the github template of .gitignore and add pycache folder

commit 9c65e3e05f5de5ca1d00311cf4ea1d513c87a8b2
Author: testno0 <##########@gmail.com>
Date:   Thu Apr 28 23:52:58 2022 +0800

	fixed the sys.path

commit d76baa8a25cfe5bd6b9856374371b581aa06bdfe
Author: testno0 <##########@gmail.com>
Date:   Thu Apr 28 23:52:24 2022 +0800

	removed function, that the script can catch the value

commit fb8697c08552eaaa4b64d796914689bf89befa9a
Author: testno0 <##########@gmail.com>
Date:   Thu Apr 28 23:51:41 2022 +0800

	fixed bash to python, it was also discovered that the systemd setup part is failing

commit dfa45f8c4c3864e3bdcc93f03622f2370daead05
Author: testno0 <##########@gmail.com>
Date:   Thu Apr 28 23:50:52 2022 +0800

	output of setup.sh

commit d7eaf989609261ab969b3da4471cee6b4a4e72cd
Author: testno0 <##########@gmail.com>
Date:   Thu Apr 28 23:35:26 2022 +0800

	removed the accidental import

commit 1c4fca9076834ba29b3b4414f410e0c13c478c0e
Author: testno0 <##########@gmail.com>
Date:   Thu Apr 28 23:35:07 2022 +0800

	print was used instead of return to store the variable in bash script

commit e2cefe80c61cfc3decb3eef9763d7d714ac776b8
Author: testno0 <##########@gmail.com>
Date:   Thu Apr 28 23:34:39 2022 +0800

	evaluators were fixed, as well as feedbacks

commit 08363d2f7e30bb587b02b74d2b619b3ce4aca75a
Author: testno0 <##########@gmail.com>
Date:   Thu Apr 28 23:21:17 2022 +0800

	replaced with update.py

commit 53a3ede30965ccc42b39f2f13e2e283f4e50b13d
Author: testno0 <##########@gmail.com>
Date:   Thu Apr 28 23:20:59 2022 +0800

	fixed the formatting

commit be3887c9a75c2d69dd723a5c39295321ef1bc9b1
Author: testno0 <##########@gmail.com>
Date:   Thu Apr 28 23:20:48 2022 +0800

	changed the sys.append

commit cad839a79ce799de8c82c02d6a13bcc0c1494c18
Author: testno0 <##########@gmail.com>
Date:   Thu Apr 28 23:20:30 2022 +0800

	removed update.sh and replaced with update.py

commit 33797d25deb59cd3691c7c7dcdad74305d9eb813
Author: testno0 <##########@gmail.com>
Date:   Thu Apr 28 22:55:59 2022 +0800

	fixed errors

commit 3ebe813fec22ff925dd15fea725121f4f283072c
Author: testno0 <##########@gmail.com>
Date:   Thu Apr 28 22:55:24 2022 +0800

	add the other dir to their path, for successful import

commit d166d28b470c4105402f3137e9c7f8fd8cb25930
Author: testno0 <##########@gmail.com>
Date:   Thu Apr 28 21:59:29 2022 +0800

	synchronized with the changes from setup.sh and code_email.py

commit a5b177f7bb614c9a4f9079c9d8166ffa35c03d21
Author: testno0 <##########@gmail.com>
Date:   Thu Apr 28 21:59:07 2022 +0800

	add prompt for school name

commit fa4020f9c9c32b47772878cfe243f48f39e7b3de
Author: testno0 <##########@gmail.com>
Date:   Thu Apr 28 21:58:41 2022 +0800

	merged the 2 email functions, and add more information in the msg

	the informations included are school name, parent, student and teachers name, as well as the time that the student logged in

commit b2f2be4bd0f12e29bf6b5b97efb0015085ddcfa7
Author: testno0 <##########@gmail.com>
Date:   Thu Apr 28 21:31:16 2022 +0800

	enforced the password access, not foolproof, can be easily wrecked by non idiots user, but since the user will be idiots, it should work

commit 0e1ac1eac3001496e280ee6116f0922426efef91
Author: testno0 <##########@gmail.com>
Date:   Thu Apr 28 21:14:34 2022 +0800

	add password generator for setup.

commit 3f1b4e3dd667900267f6eb17857da47bd15a2c4f
Author: testno0 <##########@gmail.com>
Date:   Thu Apr 28 21:14:14 2022 +0800

	initial src code, integrated the secure access

commit e54593bd77bafff86df97a59c13276c4a8d04072
Author: testno0 <##########@gmail.com>
Date:   Thu Apr 28 21:11:52 2022 +0800

	improved the accessing algorithm

	add backup, same concept with forgot password. enforced the trial limit to 3

commit b03b9d6d6b4dc4b2a069419948bb600634d0d74f
Author: testno0 <##########@gmail.com>
Date:   Thu Apr 28 20:49:14 2022 +0800

	fixed the initialized setup script

	fixed the dir of the src codes, improved the user setup, and fixed the secure setup

commit 8c2ac64cac32744c7e917345de7c497b8ff2a218
Author: testno0 <##########@gmail.com>
Date:   Thu Apr 28 20:46:27 2022 +0800

	modified the bashrc

	replaced the alias with new alias, fix PS1 to more friendly prompt, and removed the other functions

commit 754cb51a61493acc637fa1ccb6c8a5be63ef5e45
Author: testno0 <##########@gmail.com>
Date:   Thu Apr 28 20:41:28 2022 +0800

	add bashrc to the system

commit 2f37ea47eed360bc7ad8d3454421392d1400bf13
Author: testno0 <##########@gmail.com>
Date:   Thu Apr 28 19:59:24 2022 +0800

	export git log to logs file for traces and documentation

commit c3b42ef8925a84f25b0a1354ce2cf09394d0bea9
Author: testno0 <##########@gmail.com>
Date:   Thu Apr 28 19:59:08 2022 +0800

	announced the start of phase 2

commit b4f73ced94d987ae73ee54c8034f342aa16db764
Author: testno0 <##########@gmail.com>
Date:   Thu Apr 28 19:56:08 2022 +0800

	changed opencv2 to opencv-python

commit d4406378e50decf0a66fdb354a19ce2f4b7429a5
Author: testno0 <##########@gmail.com>
Date:   Thu Apr 28 19:45:49 2022 +0800

	add instructions for install

commit 64c669b4297b434e6f474a4f7909dce9df4d2a79
Author: testno0 <##########@gmail.com>
Date:   Thu Apr 28 19:43:02 2022 +0800

	improved syntaxes and formatting

commit 2b865bd3a20e3ff89eb972b9234321eb64c412e2
Author: testno0 <##########@gmail.com>
Date:   Mon Apr 25 22:37:54 2022 +0800

	improved the setup script

	changed the sequence, add more work such as full system upgrade first.

commit fee83e996df6b0e73d92c2aabca821153bb7d4ac
Author: testno0 <##########@gmail.com>
Date:   Mon Apr 25 22:33:19 2022 +0800

	moved into setup.sh

commit d50f7a20324692e0bc64d0538790425d55fa33d5
Author: testno0 <##########@gmail.com>
Date:   Mon Apr 25 22:19:34 2022 +0800

	improved exception handling, setup, and add more feedback

commit b250d101c22e37f5157eb130bb78e910aead8c03
Author: testno0 <##########@gmail.com>
Date:   Mon Apr 25 22:19:07 2022 +0800

	remove the excessive use of while loop

	the persistence was also decreased to level 3, instead of infinite

commit 76a0d462076f5094c417ef0320c15db81beb4fe8
Author: testno0 <##########@gmail.com>
Date:   Mon Apr 25 22:18:13 2022 +0800

	add alert message function

commit 9763d9f65f1f49274f2f3da65f76301bc6ee029a
Author: testno0 <##########@gmail.com>
Date:   Mon Apr 25 22:17:39 2022 +0800

	add colors for easier diagnostic or response reading

commit bb4c1a519d780d943a0816c9a6af4e2a9bdbdfc6
Author: testno0 <##########@gmail.com>
Date:   Mon Apr 25 21:51:27 2022 +0800

	created the systemd service for checking

commit e0ef3d7f2967e7815cb9d2e10266d442179e23eb
Author: testno0 <##########@gmail.com>
Date:   Mon Apr 25 21:44:26 2022 +0800

	pre update

commit 9cfff85e77d0d9725a6ef63c3879aff40b4f023c
Author: testno0 <##########@gmail.com>
Date:   Mon Apr 25 21:44:17 2022 +0800

	add secure phrase send function and msg

commit 877e1ce5c4dc2375e86170db1485a359686c40fb
Author: testno0 <##########@gmail.com>
Date:   Mon Apr 25 21:44:00 2022 +0800

	improved the codebase

	synched with function.py and simplified the codebase overall

commit 19b98211c7819d5ccd086115f3eef9ef1033b97c
Author: testno0 <##########@gmail.com>
Date:   Mon Apr 25 21:43:11 2022 +0800

	add dependency list

commit d725e452f8c6991a1a3bcdc07a82365e92390ee1
Author: testno0 <##########@gmail.com>
Date:   Mon Apr 25 21:43:01 2022 +0800

	add gitignore

commit 95e87ede62bc9a76b3f40bf7f32ce9525b6c8d01
Author: testno0 <##########@gmail.com>
Date:   Mon Apr 25 21:42:27 2022 +0800

	improved the codebase

	add secure setup function with integrated secure codephrase of 32 char

commit b603a4a7b45c7bcfc9dddd6661a326c57f2ac5be
Author: testno0 <##########@gmail.com>
Date:   Mon Apr 25 21:41:52 2022 +0800

	removed manscrpt files

commit 185770219b8b44408446247cb27bca0f6d0fadaf
Author: testno0 <96910593+testno0@users.noreply.github.com>
Date:   Mon Mar 21 15:28:38 2022 +0800

	optimized the fetch data algorithm

commit 9fe61ee33631e2fb4150a44720311e890f44a8d6
Author: testno0 <96910593+testno0@users.noreply.github.com>
Date:   Mon Mar 21 15:24:24 2022 +0800

	add information

commit 7e366a57d35774f63d6f24dea329ea1601d6824c
Author: testno0 <96910593+testno0@users.noreply.github.com>
Date:   Mon Mar 21 15:23:21 2022 +0800

	moved to /src

commit 0ee404b8f7220cc004016c7cb5eab5210dd1d94f
Author: testno0 <96910593+testno0@users.noreply.github.com>
Date:   Mon Mar 21 15:22:51 2022 +0800

	moved to /src/bin

commit 6b405ead38d2630722f54c4052bda0d600962d64
Author: testno0 <96910593+testno0@users.noreply.github.com>
Date:   Mon Mar 21 15:22:22 2022 +0800

	moved to /src/algorithm

commit 5d17b5bcf12d8cd275d08767aee2ec65b9df9966
Author: testno0 <96910593+testno0@users.noreply.github.com>
Date:   Mon Mar 21 15:21:50 2022 +0800

	moved to src

commit 5b9d5813d1d80dddc965efc81008ac645250e547
Author: testno0 <96910593+testno0@users.noreply.github.com>
Date:   Mon Mar 21 15:21:19 2022 +0800

	renamed the folder code to src

commit 2b1813b79ba12a2c7d7485cd0bb72a575bc09b06
Author: testno0 <96910593+testno0@users.noreply.github.com>
Date:   Mon Mar 21 15:20:18 2022 +0800

	Create setup.sh

commit 121a8c2bb929dcee5e15d9376e6645570c4b7156
Author: testno0 <96910593+testno0@users.noreply.github.com>
Date:   Mon Mar 21 15:20:01 2022 +0800

	Create code_email.py

commit 92f12b357c502da20119ed46eedbd071f735d172
Author: testno0 <96910593+testno0@users.noreply.github.com>
Date:   Mon Mar 21 15:16:14 2022 +0800

	Create access.py

commit 74d488c16403de4ac0edcb7ae45ad665c3d71c8e
Author: testno0 <96910593+testno0@users.noreply.github.com>
Date:   Mon Mar 14 22:29:34 2022 +0800

	upload the methodology pdf

commit 4bbdbeefb24779e8991c015055dad8eb816c2ffc
Author: testno0 <96910593+testno0@users.noreply.github.com>
Date:   Mon Mar 14 22:28:43 2022 +0800

	Create function.py

commit 3a0e00de9eb925d00514a11faa5c2a17e67e113c
Author: testno0 <96910593+testno0@users.noreply.github.com>
Date:   Mon Mar 14 22:23:31 2022 +0800

	Create main.py

commit 4e39c03dbd5bd2678a067c92f973b57213a9dc7b
Author: testno0 <96910593+testno0@users.noreply.github.com>
Date:   Mon Mar 14 22:13:56 2022 +0800

	uploaded algorithm figure

commit 785ac40abe07b87e35b5e090283611687bb271b0
Author: testno0 <96910593+testno0@users.noreply.github.com>
Date:   Mon Mar 14 22:12:35 2022 +0800

	Create methodology.tex

commit 9f76350a253e604d943f9ddeea7cbb7f094a8bb0
Author: testno0 <96910593+testno0@users.noreply.github.com>
Date:   Sun Mar 13 19:53:41 2022 +0800

	Initial commit
	
\end{lstlisting}
\doublespacing

While the command log is logged by Bourne Again SHell of GNU, from \texttt{.bash\_history}, which is exported via redirection of \texttt{stdout}.

\singlespacing
\begin{lstlisting}[caption={Command history}]
1  sudo dnf update
2  sudo rpm --import https://packages.microsoft.com/keys/microsoft.asc
3  sudo sh -c 'echo -e "[code]\nname=Visual Studio Code\nbaseurl=https://packages.microsoft.com/yumrepos/vscode\nenabled=1\ngpgcheck=1\ngpgkey=https://packages.microsoft.com/keys/microsoft.asc" > /etc/yum.repos.d/vscode.repo'
4  dnf check-update
5  sudo dnf install code
6  ls
7  git clone https://github.com/testno0/capstone
8  git clone https://github.com/iaacornus/workstation_setup
9  cd workstation_setup/
10  ls
11  cd
12  source .bashrc
13  clear
14  cd capstone/
15  git pull 
16  cd
17  git clone https://github.com/iaacornus/project_repo
18  cd
19  clear
20  git pull
21  cd capstone/
22  git pull
23  git config --global credential.helper store
24  cd
25  git clone https://github.com/iaacornus/project_repo
26  git config --global credential.helper store
27  git pull
28  cd capstone/
29  git pull
30  rm -rf capstone/
31  git clone -b devel https://github.com/testno0/capstone
32  cd capstone/
33  git status
34  git add src/algorithm/function.py 
35  git add .
36  git status
37  git commit -m "improved syntax and formatting"
38  cd
39  git config --global user.email "testno0"
40  git config --global user.name "testno0"
41  git config --global user.email "##########@gmail.com"
42  clear
43  git config user.email
44  git config user.name
45  cd capstone/p
46  cd capstone/
47  git status
48  git commit -m "improved syntaxes and formatting"
49  clear
50  git add README.md 
51  git status 
52  git commit -m "add instructions for install"
53  cd capstone/
54  ls
55  cd
56  sudo dnf install python3-pip
57  clear
58  cd capstone/
59  pip install -r requirements.txt 
60  cd
61  wget https://raw.githubusercontent.com/testno0/capstone/devel/src/algorithm/function.py
62  ls
63  cat function.py 
64  clear
65  nano .bashrc
66  source .bashrc
67  python access.py
68  python $PATH/access.py
69  access.py
70  nano .bashrc
71  source .bashrc
72  python access.py
73  access.py
74  nano .bashrc
75  source .bashrc
76  access.py
77  python access.py
78  clear
79  nano .bashrc
80  clear
81  nano .bashrc
82  source .bashrc
83  python access.py
84  nano .bashrc
85  cd src/system/
86  ls
87  mkdir utils
88  cd utils/
89  touch update.sh
90  touch init.sh
91  touch password_gen.py
92  git status
93  git add requirements.txt 
94  git commit -m "changed opencv2 to opencv-python"
95  clear
96  git history
97  git log
98  clear
99  touch logs
100  git log > logs
101  clear
102  git status
103  git add README.md 
104  git commit -m "announced the start of phase 2"
105  git add logs 
106  git commit -m "export git log to logs file for traces and documentation"
107  clear
108  touch test.py
109  rm test.py 
110  touch test.sh
111  nano test.sh 
112  bash test.sh 
113  nano test.sh 
114  bash test.sh 
115  clear
116  cd
117  git status
118  cd capstone/
119  git status
120  clear
121  rm test.sh 
122  cp ~/.bashrc ~/capstone/src/system/
123  git status
124  git add src/system/.bashrc 
125  git commit -m "add bashrc to the system"
126  echo \u
127  echo $USER
128  clear
129  git status
130  git add src/system/.bashrc 
131  git commit -m "modified the bashrc" -m "replaced the alias with new alias, fix PS1 to more friendly prompt, and removed the other functions"
132  clear
133  git status
134  git add src/system/setup.sh 
135  git commit -m "fixed the initialized setup script" -m "fixed the dir of the src codes, improved the user setup, and fixed the secure setup"
136  git status
137  clear
138  git status
139  git diff
140  clear
141  git status
142  git add src/bin/access.py 
143  git commit -m "improved the accessing algorithm" -m "add backup, same concept with forgot password. enforced the trial limit to 3"
144  git status
145  git add src/system/utils/
146  git add src/system/utils/update.sh 
147  git status
148  git restore --staged src/system/*
149  git status
150  git add src/system/utils/update.sh 
151  git commit -m "initial src code, integrated the secure access"
152  git add src/system/utils/password_gen.py 
153  git commit -m "add password generator for setup."
154  git status
155  cd
156  cd capstone/ && git pull
157  groupadd hello
158  sudo groupadd hello
159  cd
160  mkdir test
161  sudo chown :hello $HOME/test
162  sudo chmod g=r,o= $HOME/.att_sys
163  sudo chmod g=r,o= $HOME/test
164  sudo chmod g+s $HOME/.att_sys
165  sudo chmod g+s $HOME/test
166  ls
167  cat test/
168  cd test/
169  clear
170  sudo chown :hello $HOME/test
171  cd
172  sudo chown :hello $HOME/test
173  cd capstone/
174  git status
175  git diff
176  cd
177  cd test/
178  git clone https://github.com/testno0/capstone > /dev/null
179  ls
180  rm -rf capstone/
181  git clone https://github.com/testno0/capstone >> /dev/null
182  rm -rf capstone/
183  git clone https://github.com/testno0/capstone | /dev/null
184  clear
185  rm -rf capstone/
186  git clone https://github.com/testno0/capstone &>/dev/null
187  ls
188  rm -rf capstone/
189  git clone https://github.com/testno0/capstone &>/dev/null
190  rm -rf capstone/
191  git clone https://github.com/testno0/capstone &> /dev/null
192  clear
193  cd
194  ls
195  cd capstone/
196  git status
197  git commit src/system/utils/update.sh 
198  git add src/system/utils/update.sh 
199  git commit src/system/utils/update.sh 
200  git diff
201  git commit -m "enforced the password access, not foolproof, can be easily wrecked by non idiots user, but since the user will be idiots, it should work"
202  git status
203  git add src/bin/access.py 
204  git restore --staged 
205  git restore --staged src/bin/access.py
206  git status
207  git add src/bin/code_email.py 
208  git commit -m "merged the 2 email functions, and add more information in the msg" -m "the informations included are school name, parent, student and teachers name, as well as the time that the student logged in"
209  git add src/system/setup.sh 
210  git commit -m "add prompt for school name"
211  git add src/bin/access.py 
212  git commit -m "synchronized with the changes from setup.sh and code_email.py"
213  clear
214  git status
215  git push origin devel 
216  clear
217  git config --global credential.helper store
218  git pull
219  clear
220  git config --global credential.helper store
221  cd
222  git config --global credential.helper store
223  clear
224  /bin/python
225  clear
226          elif access == "student entry" 
227  ls
228  cd
229  ls
230  ls -lah
231  cd .bash
232  cd .bashrc/
233  ls
234  ls -lah
235  mv .bashrc ~
236  mv .bashrc ~/.bashrcs
237  cd
238  rm -rf .bashrc
239  mv .bashrcs .bashrc
240  source .bashrc
241  clear
242  ls
243  cd
244  rm -rf capstone/
245  git clone -b devel https://github.com/testno0/capstone
246  cd capstone/
247  ls
248  cd src/
249  ls
250  cd system/
251  ls
252  bash setup.sh 
253  clear
254  cd
255  ls
256  cd devel/capstone/
257  ls
258  cd src/system/
259  ls
260  bash setup.sh 
261  clear
262  echo pwd
263  pwd
264  ls
265  bash setup.sh 
266  clear
267  bash setup.sh 
268  clearclear
269  clear
270  bash setup.sh 
271  cd capstone/
272  ls
273  cd src/system/
274  ls
275  bash setup.sh 
276  cd
277  git pull
278  cd capstone/
279  git pull
280  cd
281  mkdir devel
282  cd devel/
283  git clone -b devel https://github.com/testno0/capstone
284  cd capstone/
285  cd
286  rm -rf capstone/
287  git clone -b devel https://github.com/testno0/capstone
288  systemctl status respository-update.service
289  cd
290  rm -rf capstone/
291  git clone -b devel https://github.com/testno0/capstone
292  cd devel/capstone/
293  ls
294  cd src/system/utils/
295  ls
296  bash update.sh 
297  cd
298  cd devel/capstone/
299  ls
300  cd src/
301  ls
302  python bin/access.py 
303  git status
304  git add .
305  git status
306  git restore --staged bin/__pycache__/code_email.cpython-310.pyc
307  git status
308  git restore --staged misc/__pycache__/*
309  git status
310  git restore *
311  git status
312  git restore --staged *
313  gits status
314  git status
315  git add algorithm/.
316  git status
317  git add bin/.
318  git commit -m "add the other dir to their path, for successful import"
319  git add system/.
320  git restore --staged *
321  git status
322  git add system/setup.sh 
323  git commit -m "fixed errors"
324  pwd
325  cd system/utils/
326  ls
327  bash update.sh 
328  pwd
329  bash update.sh 
330  clear
331  ls
332  echo
333  bash update.sh 
334  clear
335  bash update.sh 
336  pwd
337  bash update.sh 
338  python $HOME/devel/capstone/src/bin/access.py
339  ls
340  mv update.sh update.py
341  git statis
342  git status
343  git add update.
344  git commit -m "rewrite into python file, since $ret can't make interactive eval of access.py"
345  clear
346  python update.py 
347  clear
348  cd ..
349  ls
350  bash setup.sh 
351  /bin/python /home/test/devel/capstone/src/bin/access.py
352  pwd
353  cd src/misc/
354  touch __init__.py
355  /bin/python /home/test/devel/capstone/src/bin/access.py
356  clear
357  /bin/python /home/test/devel/capstone/src/bin/access.py
358  cd misc
359  ls
360  cd ..
361  cd misc/
362  touch __init__.py
363  /bin/python
364  c;ear
365  clear
366  ls
367  cd devel/capstone/
368  ls
369  cd src/
370  git status
371  git add system/utils/update.py 
372  git diff
373  git commit -m "removed update.sh and replaced with update.py"
374  git status
375  git add bin/access.py 
376  git commit -m "changed the sys.append"
377  git add system/utils/password_gen.py 
378  git commit -m "fixed the formatting"
379  git add system/utils/update.sh 
380  git commit -m "replaced with update.py"
381  git status
382  git push origin devel 
383  ls
384  bash system/setup.sh 
385  ls
386  cd system/utils/
387  ls
388  bash test.sh 
389  nano test.sh 
390  bash test.sh 
391  nano test.sh 
392  bash test.sh 
393  clear
394  bash test.sh 
395  clear
396  rm test.sh 
397  ls
398  cd ..
399  ls
400  git status
401  git add setup.sh 
402  git commit -m "evaluators were fixed, as well as feedbacks"
403  git add utils/password_gen.py 
404  git commit -m "print was used instead of return to store the variable in bash script"
405  git add utils/update.py 
406  git commit -m "removed the accidental import"
407  clear
408  git push origin devel 
409  ls
410  bash setup.sh 
411  bash $HOME/.att_sys/system/utils/update.py
412  python $HOME/.att_sys/system/utils/update.py
413  clear
414  cd
415  ls
416  cd capstone/
417  cd
418  rm -rf capstone/
419  git clone https://github.com/testno0/casptone
420  git clone -b devel https://github.com/testno0/capstone
421  cd
422  ls
423  cd capstone/
424  ls
425  cd src/system/utils/
426  ls
427  systemctl status repository-check.service
428  clear
429  cd
430  nano test.sh
431  bash test
432  bash test.sh 
433  cd
434  clear
435  ls
436  rm -rf capstone/
437  git clone -b devel https://github.com/testno0/capstone
438  rm -rf capstone/
439  git clone -b devel https://github.com/testno0/capstone
440  clear
441  /bin/python /home/test/devel/capstone/test.py
442  bash test.sh 
443  cd devel/
444  cd capstone/src/
445  ls
446  python bin/access.py 
447  clear
448  clear
449  python bin/access.py 
450  clear
451  python bin/access.py 
452  /bin/python
453  clear
454  /bin/python /home/test/devel/capstone/src/bin/access.py
455  cd devel/capstone/
456  ls
457  git status
458  status
459  clear
460  git status
461  rm .git/index
462  git status
463  git diff
464  clear
465  git pull
466  cd ..
467  rm -rf capstone/
468  git clone -b test-v1-beta https://github.com/testno0/capstone
469  cd capstone/
470  python src/bin/access.py 
471  clear
472  python src/bin/access.py 
473  cd devel/capstone/
474  python src/bin/access.py 
475  clear
476  python src/bin/access.py 
477  ls
478  python -n src/*
479  python -m src/*
480  cd src/
481  ls
482  python -m algorithm.main
483  python -m algorithm.function
484  python -m algorithm.main
485  python main.py
486  ls
487  python algorithm/main.py 
488  cd ..
489  ls
490  cd src/
491  python -m bin.code_email
492  python algorithm/main.py 
493  pwd
494  touch __init__.py
495  python algorithm/main.py 
496  python bin/access.py 
497  clear
498  python bin/access.py 
499  python algorithm/main.py 
500  clear
501  ls
502  git status
503  cd ..
504  ls
505  cd ..
506  ls
507  mkdir cli
508  cd cli/
509  git clone -b test-v1-beta https://github.com/testno0/capstone
510  clear
511  /bin/python /home/test/devel/capstone/src/bin/access.py
512  pwd
513  cd src/
514  touch bin/__init__.py
515  /bin/python /home/test/devel/capstone/src/bin/access.py
516  /bin/python /home/test/devel/capstone/src/bin/code_email.py
517  clear
518  touch misc/__init__.py
519  /bin/python /home/test/devel/capstone/src/bin/access.py
520  clear
521  /bin/python /home/test/devel/capstone/src/bin/access.py
522  clear
523  /bin/python /home/test/devel/capstone/src/bin/access.py
524  clear
525  /bin/python /home/test/devel/capstone/src/bin/access.py
526  touch algorithm/__init__.py
527  pwd
528  touch cli.py
529  rm -rf algorithm/
530  cd devel/cli/capstone/
531  ls
532  cd src/
533  cd
534  nano .bashrc
535  source .bashrc
536  cd devel/cli/capstone/
537  cd src/
538  git status
539  git diff
540  git add bin/access.py 
541  git restore --staged bin/access.py
542  git status
543  ls
544  python algorithm/main.py 
545  python algorithm/function.py 
546  gita add algorithm/function.py 
547  git add algorithm/function.py 
548  git add algorithm/main.py 
549  git status
550  git add bin/access.py 
551  git commit -m "fixed imports"
552  git add __init__.py bin/__init__.py misc/__init__.py 
553  git commit -m "this was added to allow relative imports"
554  git status
555  git add ../.gitignore 
556  git commit -m "add another dir for .gitignore"
557  git push origin test-v1-beta 
558  git config --global credential.helper store
559  clear
560  pwd
561  cd src/
562  ls
563  touch misc/__init__.py bin/__init__.py
564  touch __init__.py
565  pwd
566  cd devel/capstone/
567  git pull
568  cd ..
569  rm -rf capstone/
570  ls
571  git clone -b test-v1-beta https://github.com/testno0/capstone
572  ls
573  cd capstone/
574  ls
575  git checkout -b cli 
576  ls
577  cd src/
578  mv algorithm/* .
579  ls
580  rm -rf algorithm/
581  touch cli.py
582  touch utils.py
583  git status
584  git add cli.py 
585  git commit -m "created cli for the program"
586  git add function.py main.py 
587  git commit -m "moved to src/ instead of src/algorithm" -m "this was done for easier navigation of the application"
588  git add system/pre_setup.sh 
589  git commit -m "isolated the system setup from user setup, due to the options made in cli program" -m "since having to redo all the steps may brick the system and is too trivial and troublesome"
590  git status
591  git add system/user_setup.sh 
592  git commit -m "used for -s option in cli.py"
593  git add system/.bashrc 
594  git commit -m "removed the .bashrc, since alias aren't needed anymore as the options were unified in cli.py"
595  git add system/utils/update.py 
596  git commit -m "turned into function, so can be easily called from cli.py"
597  git status
598  git add algorithm/* 
599  git add algorithm
600  git status
601  git commit -m "moved into src/"
602  git add system/setup.sh 
603  git commit -m "splited into user_setup.sh and pre_setup.sh"
604  git status
605  git push origin cli 
606  git diff
607  git add system/user_setup.sh 
608  git commit -m "moved the user setup part from setup.sh"
609  ls
610  python cli.py -u
611  python cli.py
612  clear
613  python cli.py
614  clear
615  python cli.py
616  clear
617  python cli.py 0u
618  python cli.py -u
619  python cli.py -U
620  /bin/python /home/test/devel/capstone/src/main.py
621  cd devel/capstone/src/
622  ls
623  python cli.py -U
624  cd devel/capstone/src/
625  clear
626  python cli.py --update
627  clear
628  python cli.py --setup
629  git status
630  git add system/.bashrc 
631  git commit -m "aliased python cli.py to taptap instead of creating a setup.py script"
632  cd ..
633  ls
634  rm -rf tests/
635  git status
636  git add tests 
637  git status
638  git commit -m "removed tests folder since it is nonsensical"
639  git status
640  git rm --cached */__pycache__/*
641  git status
642  git ad .gitignore
643  git add .gitignore
644  git restore --staged .gitignore
645  git add .gitignore
646  git commit -m "removed other entries"
647  git status
648  git rm src/bin/__pycache__/code_email.cpython-310.pyc
649  git rm -f src/bin/__pycache__/code_email.cpython-310.pyc
650  git status
651  git restore --staged src/bin/__pycache__/
652  git status
653  git diff
654  git add src/system/utils/update.py 
655  git commit -m "fixed imports and conditions"
656  git status
657  git add src/function.py 
658  git commit -m "fixed the setup() function" -m "other information were instead retrieved from the user file info instead of being a parameter"
659  git diff
660  git add src/bin/access.py 
661  git commit -m "fixed the output as well as exception handling, and variable ref"
662  git status
663  clear
664  git push origin cli 
665  python cli.py -U
666  cd src/
667  python cli.py -U
668  python cli.py -u
669  python cli.py -h
670  cd ..
671  git log
672  git log > logs
673  git status
674  git add logs 
675  git commit -m "reexproted the logs as of may 01, 2022 @ 20:48"
676  git add README.md 
677  git commit -m "add info about merging plan, and documentation of cli"
678  clear
679  git push origin cli 
680  clear
681  git status
682  git diff
683  git status
684  cd src/
685  git add system/.bashrc 
686  git commit -m "changed the dir of alias"
687  git add system/user_setup.sh 
688  git commit -m "used > redirection to avoid errors"
689  git add system/pre_setup.sh 
690  git commit -m "fixed the system structure and add chmod to the script to easily execute the script"
691  git status
692  git add bin/code_email.py 
693  git commit -m "changed the message if the student is present"
694  clear
695  git diff
696  python cli.py --use
697  cd
698  ls
699  cd .att_sys/
700  ls
701  cat user_info 
702  cd
703  cp .bashrc ~/devel/capstone/src/system/
704  git status
705  cd devel/capstone/src/
706  clear
707  git status
708  git diff
709  python cli.py --use
710  cd
711  cat .att_sys/user_info 
712  cd devel/capstone/src/
713  python cli.py --use
714  /bin/python /home/test/devel/capstone/src/main.py
715  cd
716  cat .att_sys/user_info 
717  clear
718  python cli.py --destroy
719  git status
720  git add cli.py 
721  git diff
722  git commit -m "tested and add commands"
723  git status
724  git add main.py 
725  git status
726  git commit -m "satisfied the requirements of imports from code_email and function.py"
727  git add function.py 
728  git commit -m "the loop bug were removed, the trial limit were also barred into 3" -m "the feedback were also changed"
729  git push origin cli 
730  ls ~
731  ls ~ | less
732  less < ls `
733  less < ls ~
734  less < $(ls ~)
735  less test.sh
736  cat > test.sh
737  cat < test.sh > hello.txt
738  cat hello.txt 
739  cat test
740  cat test.sh 
741  echo "hello"" > test.sh
742  echo "hello" > test.sh
743  cat test.sh 
744  cat < test.sh > hello.txt
745  cat hello.txt 
746  cat test.sh | less
747  less < cat test.sh
748  ls
749  ls | tee hello.txt 
750  cat hello.txt 
751  env
752  echo 'The quick brown; fox jumps over the lazy  dog' > sample.txt
753  cut -c 5 sample.txt 
754  cut -f 2 sample.txt
755  cut -c -5 sample.txt
756  cut -c 5- sample.txt
757  cut -f 2 sample.txt
758  nano sample.txt 
759  cut -f 2 sample.txt
760  cut -f 1 -d ";" sample.txt
761  nano sample.txt 
762  cut -c 1 sample.txt 
763  cut -c 1-10 sample.txt 
764  journalctl -xb
765  journalctl -xb | head
766  clear
767  journalctl -xb | head
768  clear
769  journalctl -xb | tail -f
770  tail -f /var/log/syslog
771  journalctl -xb > log.txt
772  tail -f log.txt 
773  pip install face_recognition
774  sudo dnf install cmake
775  pip install face_recognition
776  cmake
777  pip3 install face_recognition
778  sudo dnf install python3-dlib
779  pip3 install face_recognition
780  dnf list installed | grep -i "python.dlib"
781  dnf list installed | grep -i "python3-dlib"
782  dnf list installed | grep -i "python.-dlib"
783  cp -h
784  cp --help
785  cp --help | grep -i "-r"
786  cp --help | grep -i "r"
787  clear
788  cp --help | grep -i "r"
789  clear
790  mkdir .att_sys
791  cd devel/capstone/
792  status
793  git status
794  add requirements.txt 
795  git add requirements.txt 
796  git commit -m "add face_recognition as requirements and removed cv2"
797  add src/system/pre_setup.sh 
798  git add src/system/pre_setup.sh 
799  git commit -m "simplified the src code moving into the app folder"
800  clear
801  git log > logs
802  status
803  git status
804  git add logs 
805  diff
806  git diff
807  clear
808  git commit -m "add the new commits"
809  clear
810  history > command_logs
159  lsblk
160  sudo wipefs -a /dev/sdc
161  sudo umount /dev/sdc
162  sudo umount /dev/sdc1
163  sudo wipefs -a /dev/sdc
164  lsblk
165  sudo gdisk /dev/sdc
166  clear
167  sudo mkfs.ext4 /dev/sdc1
168  cd temporary/
169  git clone -b face_recog https://github.com/testno0/capstone
170  cd temporary/capstone/
171  touch rfid/main2.cpp
172  flatpak install flathub org.gnome.Cheese
173  sudo dnf install cmake
174  souce venv/bin/activate
175  source venv/bin/activate
176  pip install -r requirements.txt 
177  sudo dnf install cmake
178  sudo dnf install python3-dlib
179  pip install -r requirements.txt 
180  sudo dnf install gcc-c++
181  clear
182  pip install -r requirements.txt 
183  clear
184  sudo dnf install openbla
185  sudo dnf install openblas
186  pip install -r requirements.txt 
187  sudo pip install -r requirements.txt 
188  sudo pip3 install -r requirements.txt 
189  pip install
190  sudo pip install -r requirements.txt 
191  sudo python3-pip install -r requirements.txt 
192  clear
193  exit
194  cd temporary/capstone/
195  clear
196  source venv/bin/activate
197  clear
198  pip install -r requirements.txt 
199  deactivate
200  toolbox enter
201  exit
202  source venv/bin/activate
203  clear
204  ssh icq-relationship-photographic-knew.trycloudflare.com
205  exit
206  cd temporary/capstone/
207  pull
208  ls
209  python -m venv venv
210  source venv/bin/activate
211  clear
212  deactivate
213  toolbox enter
214  htop
215  exit
216  source venv/bin/activate
217  clear
218  pip install -r requirements.txt 
219  clear
220  python src/tests/fr_test.py 
221  clear
222  pip install face_recognition
223  python -m pip install --upgrade pip
224  pip install face_recognition
225  sudo dnf install python-devel
226  pip install face_recognition
227  clear
228  pip install -r requirements.txt 
229  clear
230  python src/tests/fr_test.py 
231  python src/tests/fr_test2.py 
232  python src/tests/fr_test.py 
233  pip install python-cv2
234  pip install opencv2-python
235  pip install opencv-python
236  deactivate
237  exit
238  clear
239  cd temporary/capstone/
240  clear
241  toolbox enter
242  clear
243  cd temporary/capstone/
244  status
245  commit -m "fixed the while loop"
246  source /var/home/iaacornus/temporary/capstone/venv/bin/activate
247  /var/home/iaacornus/temporary/capstone/venv/bin/python /var/home/iaacornus/temporary/capstone/src/tests/fr_test.py
248  toolbox enter
249  deactivate
250  toolbox enter
251  htop
252  cat .git-credentials 
253  git config --global commit.gpgsign true
254  cd Development/workstation_setup/
255  ls
256  nano silverblue_setup.md 
257  status
258  add silverblue_setup.md 
259  commit -m "little changes"
260  git config user.name
261  clear
262  git config user.signingkey
263  commit -m "little changes"
264  killall gpg-agent
265  commit -m "little changes"
266  clear
267  git config --global rebase.autoStash true
268  clear
269  cd temporary/capstone/
270  status
271  commit -m "fixed while loops"
272  add src/bin/*
273  status
274  commit -m "face recognition pycache"
275  add src/face_recog.py src/tests/fr_test.py 
276  commit -m "successful face recognition build"
277  clear
278  add sample/
279  commit -m "face sample for testing"
280  status
281  add rfid/main2.cpp 
282  commit -m "another rfid build"
283  diff
284  add src/function.py 
285  commit -m "add av_cam function to simplify face_recognition.py"
286  clear
287  push origin face_recog
288  pull face_recog https://github.com/testno0/capstoen
289  pull face_recog https://github.com/testno0/capstone
290  pull
291  push origin face_recog
292  git pull
293  git rebase
294  git pull
295  git rebasels
296  rebase face_recog
297  lbranch
298  git checkout face_recog
299  ls
300  ls rfid/
301  git stash
302  git reset --merge
303  git stash
304  git checkout face_recog
305  status
306  push origin face_recog
307  pull
308  pull face_recog
309  pull https://github.com/testno0/capstone face_recog
310  push
311  git pull --rebase
312  git pull --rebase origin face_recog
313  rm -fr ".git/rebase-merge"
314  git pull --rebase origin face_recog
315  push origin main
316  clear
317  status
318  git rebase --skip
319  status
320  push origin face_recog
321  history
322  sudo dnf clean all
323  sudo dnf autoremove
324  clear
325  exit
326  toolbox enter
327  history | grep install
328  flatpak list
329  flatpak uninstall --system --delete-data org.gnome.Cheese
330  flatpak uninstall --unused
331  flatpak repair
332  clear
333  toolbox enter
334  toolbox list
335  toolbox rmi fedora-toolbox-35
336  toolbox rm -a
337  killall toolbox
338  toolbox rm -a
339  toolbox list
340  toolbox rm -a
341  toolbox rm fedora-toolbox-35
342  cd temporary/capstone/
343  ls
344  log
345  log >> logs 
346  log > logs 
347  history >> command_logs 
\end{lstlisting}
\doublespacing

The source codes are verifiable using \texttt{sha256sum} command:

\singlespacing
\begin{lstlisting}[language=bash, caption={Checking of cryptographic hash of source codes before use.}]
git clone https://github.com/testno0/capstone
tar -cvzf capstone/
echo -e "shasum=$(sha256sum capstone.tar*)\nif [[ $shasum = <shasum> ]]; then\n\techo '> Passed.'\nelse\n\techo '> Shasum doesn't match.'\nfi" > test.sh
chmod +x test.sh
./test.sh
\end{lstlisting}
	
\end{document}